%%
%% This is file `sample-sigconf-i13n.tex',
%% generated with the docstrip utility.
%%
%% The original source files were:
%%
%% samples.dtx  (with options: `sigconf-i13n')
%% 
%% IMPORTANT NOTICE:
%% 
%% For the copyright see the source file.
%% 
%% Any modified versions of this file must be renamed
%% with new filenames distinct from sample-sigconf-i13n.tex.
%% 
%% For distribution of the original source see the terms
%% for copying and modification in the file samples.dtx.
%% 
%% This generated file may be distributed as long as the
%% original source files, as listed above, are part of the
%% same distribution. (The sources need not necessarily be
%% in the same archive or directory.)
%%
%%
%% Commands for TeXCount
%TC:macro \cite [option:text,text]
%TC:macro \citep [option:text,text]
%TC:macro \citet [option:text,text]
%TC:envir table 0 1
%TC:envir table* 0 1
%TC:envir tabular [ignore] word
%TC:envir displaymath 0 word
%TC:envir math 0 word
%TC:envir comment 0 0
%%
%%
%% The first command in your LaTeX source must be the \documentclass command.
\documentclass[a4paper,sigconf, language=french,
language=german, language=spanish, language=english]{acmart}

%% \BibTeX command to typeset BibTeX logo in the docs
\AtBeginDocument{%
  \providecommand\BibTeX{{%
    \normalfont B\kern-0.5em{\scshape i\kern-0.25em b}\kern-0.8em\TeX}}}

%% Rights management information.  This information is sent to you
%% when you complete the rights form.  These commands have SAMPLE
%% values in them; it is your responsibility as an author to replace
%% the commands and values with those provided to you when you
%% complete the rights form.
\setcopyright{rightsretained}
\copyrightyear{2022}
\acmYear{2022}
\acmDOI{}

%% These commands are for a PROCEEDINGS abstract or paper.

\acmConference[Colloquium on Vulnerability Discovery 2022]{Colloquium on Vulnerability Discovery 2022}{August 1st, 2022}{KIT, Karlsruhe, Germany}
\acmPrice{00.00}
\acmISBN{}

\settopmatter{printacmref=false}

\settopmatter{printccs=false}

%%
%% Submission ID.
%% Use this when submitting an article to a sponsored event. You'll
%% receive a unique submission ID from the organizers
%% of the event, and this ID should be used as the parameter to this command.
%%\acmSubmissionID{123-A56-BU3}

%%
%% For managing citations, it is recommended to use bibliography
%% files in BibTeX format.
%%
%% You can then either use BibTeX with the ACM-Reference-Format style,
%% or BibLaTeX with the acmnumeric or acmauthoryear sytles, that include
%% support for advanced citation of software artefact from the
%% biblatex-software package, also separately available on CTAN.
%%
%% Look at the sample-*-biblatex.tex files for templates showcasing
%% the biblatex styles.
%%

%%
%% The majority of ACM publications use numbered citations and
%% references.  The command \citestyle{authoryear} switches to the
%% "author year" style.
%%
%% If you are preparing content for an event
%% sponsored by ACM SIGGRAPH, you must use the "author year" style of
%% citations and references.
%% Uncommenting
%% the next command will enable that style.
%%\citestyle{acmauthoryear}

%%
%% end of the preamble, start of the body of the document source.
% Copyright 2017 Sergei Tikhomirov, MIT License
% https://github.com/s-tikhomirov/solidity-latex-highlighting/

\usepackage{listings, xcolor}

\definecolor{verylightgray}{rgb}{.97,.97,.97}

\lstdefinelanguage{Solidity}{
	keywords=[1]{anonymous, assembly, assert, balance, break, call, callcode, case, catch, class, constant, continue, constructor, contract, debugger, default, delegatecall, delete, do, else, emit, event, experimental, export, external, false, finally, for, function, gas, if, implements, import, in, indexed, instanceof, interface, internal, is, length, library, log0, log1, log2, log3, log4, memory, modifier, new, payable, pragma, private, protected, public, pure, push, require, return, returns, revert, selfdestruct, send, solidity, storage, struct, suicide, super, switch, then, this, throw, transfer, true, try, typeof, using, value, view, while, with, addmod, ecrecover, keccak256, mulmod, ripemd160, sha256, sha3}, % generic keywords including crypto operations
	keywordstyle=[1]\color{blue}\bfseries,
	keywords=[2]{address, bool, byte, bytes, bytes1, bytes2, bytes3, bytes4, bytes5, bytes6, bytes7, bytes8, bytes9, bytes10, bytes11, bytes12, bytes13, bytes14, bytes15, bytes16, bytes17, bytes18, bytes19, bytes20, bytes21, bytes22, bytes23, bytes24, bytes25, bytes26, bytes27, bytes28, bytes29, bytes30, bytes31, bytes32, enum, int, int8, int16, int24, int32, int40, int48, int56, int64, int72, int80, int88, int96, int104, int112, int120, int128, int136, int144, int152, int160, int168, int176, int184, int192, int200, int208, int216, int224, int232, int240, int248, int256, mapping, string, uint, uint8, uint16, uint24, uint32, uint40, uint48, uint56, uint64, uint72, uint80, uint88, uint96, uint104, uint112, uint120, uint128, uint136, uint144, uint152, uint160, uint168, uint176, uint184, uint192, uint200, uint208, uint216, uint224, uint232, uint240, uint248, uint256, var, void, ether, finney, szabo, wei, days, hours, minutes, seconds, weeks, years},	% types; money and time units
	keywordstyle=[2]\color{teal}\bfseries,
	keywords=[3]{block, blockhash, coinbase, difficulty, gaslimit, number, timestamp, msg, data, gas, sender, sig, value, now, tx, gasprice, origin},	% environment variables
	keywordstyle=[3]\color{violet}\bfseries,
	identifierstyle=\color{black},
	sensitive=false,
	comment=[l]{//},
	morecomment=[s]{/*}{*/},
	commentstyle=\color{gray}\ttfamily,
	stringstyle=\color{red}\ttfamily,
	morestring=[b]',
	morestring=[b]"
}

\lstset{
	language=Solidity,
	backgroundcolor=\color{verylightgray},
	extendedchars=true,
	basicstyle=\footnotesize\ttfamily,
	showstringspaces=false,
	showspaces=false,
	numbers=left,
	numberstyle=\footnotesize,
	numbersep=9pt,
	tabsize=2,
	breaklines=true,
	showtabs=false,
	captionpos=b
}
	
\begin{document}

%%
%% The "title" command has an optional parameter,
%% allowing the author to define a "short title" to be used in page headers.
\title{Finding Vulnerabilities in Smart Contracts}


%%
%% The "author" command and its associated commands are used to define
%% the authors and their affiliations.
%% Of note is the shared affiliation of the first two authors, and the
%% "authornote" and "authornotemark" commands
%% used to denote shared contribution to the research.
\author{Michele Massetti}
\email{big.michelemassetti@gmail.com}
\affiliation{%
  \institution{Karlsruhe Institute of Technologies}
  \city{Karlsruhe}
  \state{Baden-Württemberg}
  \country{Germany}
}

%%
%% By default, the full list of authors will be used in the page
%% headers. Often, this list is too long, and will overlap
%% other information printed in the page headers. This command allows
%% the author to define a more concise list
%% of authors' names for this purpose.
\renewcommand{\shortauthors}{Massetti.}

%%
%% The abstract is a short summary of the work to be presented in the
%% article.
\begin{abstract}
Blockchain is a revolutionary technology that enables users to communicate in a trust-less manner. 
The most prominent change brought by this technology is the mode of business between organizations: they do not need anymore a trusted third party. 
It is a distributed ledger technology based on a decentralized peer-to-peer (P2P) network. 
Since Bitcoin was deployed, many blockchain systems have been born with more capabilities, which have allowed them to fit many
different use cases. Smart Contracts, which are programs running on blockchain systems, could extend the potentiality of blockchain
from a platform for financial transactions to an all-purpose utility.
The development of innovative and prominent applications is a consequence of them, such as NFT marketplaces, music royalty tracking, supply chain and logistics monitoring, voting mechanism, 
cross-border payments, and many others.
Finding bugs and vulnerabilities in them is necessary for assuring their correct behaviour. 
This paper deals with the way for finding the vulnerabilities in Ethereum blockchain-based smart contracts. We review related works regarding 
the classification of the most common vulnerabilities and tools which support their detection.
\end{abstract}

%%
%% The code below is generated by the tool at http://dl.acm.org/ccs.cfm.
%% Please copy and paste the code instead of the example below.
%%
\begin{CCSXML}
<ccs2012>
 <concept>
  <concept_id>10010520.10010553.10010562</concept_id>
  <concept_desc>Computer systems organization~Embedded systems</concept_desc>
  <concept_significance>500</concept_significance>
 </concept>
 <concept>
  <concept_id>10010520.10010575.10010755</concept_id>
  <concept_desc>Computer systems organization~Redundancy</concept_desc>
  <concept_significance>300</concept_significance>
 </concept>
 <concept>
  <concept_id>10010520.10010553.10010554</concept_id>
  <concept_desc>Computer systems organization~Robotics</concept_desc>
  <concept_significance>100</concept_significance>
 </concept>
 <concept>
  <concept_id>10003033.10003083.10003095</concept_id>
  <concept_desc>Networks~Network reliability</concept_desc>
  <concept_significance>100</concept_significance>
 </concept>
</ccs2012>
\end{CCSXML}

\ccsdesc[500]{Computer systems organization~Embedded systems}
\ccsdesc[300]{Computer systems organization~Redundancy}
\ccsdesc{Computer systems organization~Robotics}
\ccsdesc[100]{Networks~Network reliability}

%%
%% Keywords. The author(s) should pick words that accurately describe
%% the work being presented. Separate the keywords with commas.
\keywords{Solidity, Software, Vulnerability, Blockchain}


%% A "teaser" image appears between the author and affiliation
%% information and the body of the document, and typically spans the
%% page.


%%
%% This command processes the author and affiliation and title
%% information and builds the first part of the formatted document.
\maketitle

\section{Introduction}
Nowadays, the major platform for decentralized decentralized finance (DeFi) 
and applications (dApps) is Ethereum. It can be described as the "internet of Blockchain".
Its ecosystem consists of the underlying blockchain, a large
variety of smart contracts deployed on it, a wide range of
valuable assets. 

This growing technology has attaracted many investors, indeed, according \href{https://www.coingecko.com/}{CoinGeko}, 
the crypto market's value is standing around \$2 trillion.
On the other hand, interest in such a market has grown even among malicious attackers. 
Attacks such as the “Parity Wallet Hack” and the “Decentralized Autonomous Organization Attack” cost millions of dollars simply because of 
naive bugs in the smart contract code. Blockchain and smart contract technologies have multiple aims, but unfortunately, new applications 
based on them still contain bugs and multiple vulnerabilities, which cause 
several issues for the end-users. Most of the use of this technology relates to finance or certifications, therefore integrity, 
authentication and authorisation in transactions are mandatory. 

The research field behind blockchain technology is growing, as well as the one concerning 
its security and accordingly, many analysis tools were developed. 
These incorporate various strategies for performing the analyses, concerning the technical aspects of smart contracts, 
so these would work differently according to the object of the analysis. 

Among the many aspects of smart contract, our systematic
literature review focuses on studies related to vulnerabilities and 
analysis tools for their detection. We will try to give an answer to the following 
research questions:
\begin{itemize}
    \item Which are the main vulnerabilities in Smart Contracts?
    \item Which methodologies are implemented by analysis tools?
    \item How should we behave for the detection of Vulnerabilities?
\end{itemize}

In Section \ref{RelatedWorks}, we compare the actual papers and works regarding this topic. 
Section \ref{Vulnerabilities} explains the objective of our analysis: Smart Contracts Vulnerabilities. 
We give a taxonomy for the main vulnerabilities regarding Solidity. The classification of analysis tools 
is shown in Section \ref{Tools}. We discuss about the main strategies implemented by those.
Options
\begin{itemize}
  \item In the last Section \ref{Conclusion} we try to define a guideline for detecting vulnerabilities. 
  \item In the last Section \ref{Conclusion} we give an overview, defining the suitable cases which 
the tools work better.
  \item In the last section \ref{Conclusion} we propose a real case.
\end{itemize}


\section{Releted Works}
\label{RelatedWorks}
\begin{itemize}
  \item Papers about vulenrabilities detection for defing a taxonomy
  \item papers regarding comperison between tools
  \item papers of the tools that we want to have a Look
\end{itemize}

\citet{SystematicReviewVuln} 

\section{Vulnerabilities in Solidity}
\label{Vulnerabilities}
An overview of classification of Vulnerabilities in Solidity is given in this section. 
Classifications by scholars and community taxonomies are the strating point for drowing our own classification. 
These knoledges are used for having our own classification; we depicted which ones are the most relevant 
vulnerabilities for our work.
The term vulnerability is used in a similar sense than is common in computer security. 
It refers to a weakness or limitation of a smart contract that may result in
security problems. 
A vulnerability represents even the way the attacker can exploit the contract. 
This includes locked or stolen
resources, breaches of confidentiality or data integrity, and state
changes in the environment of smart contracts that were not
detected by developers or users, therefor they possibly be exploited by a malicious part.
\subsection{Solidity Vulnerabilities and real-world examples}
\label{sec:Vulenerabilities:Definifinition}

\subsection{Vulnerabilities Classification and examples}
\label{sec:Vulnerabilities:Classification}
We summeries 10 different Vulnerabilities, providing a definition and a real-world example.
\paragraph{Reentrancy} It is one of the vulnerabilities much more exploited during the recent year. 
Many attacks occured and could steal great amount of many to vulnerable protocol. 
A reentrancy attack can occur when a function makes an external call to another untrusted contract 
before it resolves any effects. 
A recursive call back to the original function is made by the attacker, before 
the function could updated the state of the state of the variables. 
The goal is repeating interactions that would have otherwise not run after the effects were resolved. 
The function withdaw, line 8 to 16 \autoref{lst:Reentrancy}, implements this bug, because the balance of the user is updated after the call. 
The attacker can run multiple the function withdaw without updating the balance, withrawing all the funds.
\begin{lstlisting} [language=Solidity, caption={Reentrancy}, label={lst:Reentrancy}]
contract EtherStore {
    mapping(address => uint) public balances;
    ...
    //vulnerable function exploited by the attacker
    function withdraw() public {
        uint bal = balances[msg.sender];
        require(bal > 0);

        (bool sent, ) = msg.sender.call{value: bal}("");
        require(sent, "Failed to send Ether");

        balances[msg.sender] = 0;
    }
    ...
}
  
\end{lstlisting}
\paragraph{Unexpected Ether balance} An exploites can generated a Dos attack when the smart contract strictly assume a specific token balance. 
It is always possible to forcibly send ether to a contract (without triggering its fallback function), using selfdestruct, or by mining to the account. 
A common practice for detecting this bug is checking the invariant. 
Smart contracts which manage ERC20token, currenciens built on the blockchain, keep track of the total supplay. 
This is an example of invariant.

\paragraph{Delegatecall to Untrusted Callee} The definition of delegatecall is given by \cite{SolDocs}. 
Delegatecall is a special variant of a message call. 
It is identical to a message call apart from the fact that the code at the target address is executed in the context of the calling contract and msg.sender and msg.value do not change their values.
As a result of the context-preserving nature of DELEGATECALL, building vulnerability-free custom libraries isn't as easy as one might think. 
The code in libraries themselves can be secure and vulnerability-free; however, when run within the context of another application new vulnerabilities can arise.
An example is the use of a function call to send Ether, which could cause the execution of the
fallback function of a malicious attacker. The consequence can be a reentrancy attack.
\paragraph{Overflows \& Underflows} The Ethereum Virtual Machine (EVM) defineds fixed-size data types for integers. 
An example is the identifier uint8, unsign integer over 8 bit, for variables means the variable can only store numbers in the range [0,255].
The attacker can craft the input value of a vulnerable contract's function in order to force a specific operation. 
In the contract TimeLock in \autoref{lst:ArithmeticOp}, an attacker could forge a parameter of the function function increaseLockTime, 
in oder to set lockTime to 0 because of Overflows. So it can call withraw and passing the require at line 17.
\begin{lstlisting}[language=Solidity,caption={Overflows \& Underflows},label={lst:ArithmeticOp}]
contract TimeLock {
  
    mapping(address => uint) public balances;
    mapping(address => uint) public lockTime;
    
    function deposit() public payable {
        balances[msg.sender] += msg.value;
        lockTime[msg.sender] = now + 1 weeks;
    }
    
    function increaseLockTime(uint _secondsToIncrease) public {
        lockTime[msg.sender] += _secondsToIncrease;
    }
    
    function withdraw() public {
        require(balances[msg.sender] > 0);
        require(now > lockTime[msg.sender]);
        msg.sender.transfer(balances[msg.sender]);
        balances[msg.sender] = 0;
    }
}
\end{lstlisting}
\paragraph{Authentication \& Access Control Vulnerabilities} 
\paragraph{Transaction Order Dependence}
\paragraph{Block Timestamp Manipulation}
\paragraph{Unchecked CALL Return Values}
\paragraph{Race Conditions / Front Running}
\paragraph{Denial Of Service (DOS)}{rthyj}
\\
The \autoref{tab:classification} gives a graphical overview of our classification, referring with the SWC Registry and DASP.

\begin{center}
\begin{table} 
\caption{Our Classification of Vulnerabilities} 
  \begin{tabular}{c c c} 
    \hline
      Vulnerability & DASP & SWC \\ [0.5ex] 

    \hline\hline
    Reentrancy & 1 & 107
    
  \end{tabular}
  \label{tab:classification}
\end{table}
\end{center}

\section{Methodologies implemented by security analysis tools}
\label{MethodologiesForTools}
\paragraph{Static analysis} refers to a class of methods that examine
the source code or bytecode of a contract without execut-
ing it. Most methods listed below are static.
\paragraph{Dynamic analysis} means to observe a contract while
executing (parts of) it in the original context.
\paragraph{Disassembling} means to translate EVM bytecode into bet-
ter readable assembly language, where machine operations
and storage addresses are represented symbolically.
\paragraph{Decompilation} is the process of transforming EVM byte-
code to a more compact representation on a higher abstrac-
tion level (like intermediate or Solidity code) to enhance
the readability of the code or to ease data flow analysis.
\paragraph{Control flow graph (CFG)} is a directed graph, where the
basic blocks of a program serve as the nodes. An arc
connects node A with node B if it is possible that block B
gets executed immediately after block A. The arc may be
labeled by the condition under which this path is chosen.
\paragraph{Formal verification} means verification by formal methods
with the aim of proving or disproving system properties
rigorously. As a prerequisite, all components referenced
by such a property as well as their behavior must have
been specified formally. E.g., to verify properties of smart
contracts on bytecode level formally, we need a formal
specification of the EVM and of the properties.
\paragraph{Dynamic CFG} is similar to a CFG with the difference that
arcs indicate the actual control flow encountered during a
particular execution of the code.
\paragraph{Model checking} is a technique for automatically verifying
correctness properties of finite-state systems. It requires a
model of the system which is then checked against a given
specification.
\paragraph{Call graph} is a directed graph, where the nodes are
functions. There is an arc from node A to node B if
function A calls function B.
\paragraph{Abstract Syntax Tree (AST)}represents the syntactic struc-
ture of Solidity code as a tree. It occurs as an intermediate
product when compiling Solidity to bytecode. Often, it is
better suited for analyzing Solidity code.
\paragraph{Symbolic execution} means to execute code using symbols
instead of concrete values for the variables. Operations
on these symbols lead to algebraic terms, and conditional
statements give rise to propositional formulas that char-
acterize the branches. A particular part of the code is
reachable if the conjunction of formulas on the path to this
part is satisfiable, which can be checked by SMT-solvers.

\section{Security Analysis Tools}
\label{Tools}


\section{Real-World Case}
\label{RealCase}

\section{Conclusion}
\label{Conclusion}


\begin{acks}

\end{acks}

%%
%% The next two lines define the bibliography style to be used, and
%% the bibliography file.
\bibliographystyle{ACM-Reference-Format}
\bibliography{sample-base}

%%
%% If your work has an appendix, this is the place to put it.
\appendix

\section{Research Methods}

\subsection{Part One}


\end{document}
\endinput
%%
%% End of file `sample-sigconf-i13n.tex'.
