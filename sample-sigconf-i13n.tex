%%
%% This is file `sample-sigconf-i13n.tex',
%% generated with the docstrip utility.
%%
%% The original source files were:
%%
%% samples.dtx  (with options: `sigconf-i13n')
%% 
%% IMPORTANT NOTICE:
%% 
%% For the copyright see the source file.
%% 
%% Any modified versions of this file must be renamed
%% with new filenames distinct from sample-sigconf-i13n.tex.
%% 
%% For distribution of the original source see the terms
%% for copying and modification in the file samples.dtx.
%% 
%% This generated file may be distributed as long as the
%% original source files, as listed above, are part of the
%% same distribution. (The sources need not necessarily be
%% in the same archive or directory.)
%%
%%
%% Commands for TeXCount
%TC:macro \cite [option:text,text]
%TC:macro \citep [option:text,text]
%TC:macro \citet [option:text,text]
%TC:envir table 0 1
%TC:envir table* 0 1
%TC:envir tabular [ignore] word
%TC:envir displaymath 0 word
%TC:envir math 0 word
%TC:envir comment 0 0
%%
%%
%% The first command in your LaTeX source must be the \documentclass command.
\documentclass[a4paper,sigconf, language=french,
language=german, language=spanish, language=english]{acmart}

%% \BibTeX command to typeset BibTeX logo in the docs
\AtBeginDocument{%
  \providecommand\BibTeX{{%
    \normalfont B\kern-0.5em{\scshape i\kern-0.25em b}\kern-0.8em\TeX}}}

%% Rights management information.  This information is sent to you
%% when you complete the rights form.  These commands have SAMPLE
%% values in them; it is your responsibility as an author to replace
%% the commands and values with those provided to you when you
%% complete the rights form.
\setcopyright{rightsretained}
\copyrightyear{2022}
\acmYear{2022}
\acmDOI{}

%% These commands are for a PROCEEDINGS abstract or paper.

\acmConference[Colloquium on Vulnerability Discovery 2022]{Colloquium on Vulnerability Discovery 2022}{August 1st, 2022}{KIT, Karlsruhe, Germany}
\acmPrice{00.00}
\acmISBN{}

\settopmatter{printacmref=false}

\settopmatter{printccs=false}

%%
%% Submission ID.
%% Use this when submitting an article to a sponsored event. You'll
%% receive a unique submission ID from the organizers
%% of the event, and this ID should be used as the parameter to this command.
%%\acmSubmissionID{123-A56-BU3}

%%
%% For managing citations, it is recommended to use bibliography
%% files in BibTeX format.
%%
%% You can then either use BibTeX with the ACM-Reference-Format style,
%% or BibLaTeX with the acmnumeric or acmauthoryear sytles, that include
%% support for advanced citation of software artefact from the
%% biblatex-software package, also separately available on CTAN.
%%
%% Look at the sample-*-biblatex.tex files for templates showcasing
%% the biblatex styles.
%%

%%
%% The majority of ACM publications use numbered citations and
%% references.  The command \citestyle{authoryear} switches to the
%% "author year" style.
%%
%% If you are preparing content for an event
%% sponsored by ACM SIGGRAPH, you must use the "author year" style of
%% citations and references.
%% Uncommenting
%% the next command will enable that style.
%%\citestyle{acmauthoryear}

%%
%% end of the preamble, start of the body of the document source.
\begin{document}

%%
%% The "title" command has an optional parameter,
%% allowing the author to define a "short title" to be used in page headers.
\title{Finding Vulnerabilities in Smart Contracts}


%%
%% The "author" command and its associated commands are used to define
%% the authors and their affiliations.
%% Of note is the shared affiliation of the first two authors, and the
%% "authornote" and "authornotemark" commands
%% used to denote shared contribution to the research.
\author{Michele Massetti}
\email{big.michelemassetti@gmail.com}
\affiliation{%
  \institution{Karlsruhe Institute of Technologies}
  \city{Karlsruhe}
  \state{Baden-Württemberg}
  \country{Germany}
}

%%
%% By default, the full list of authors will be used in the page
%% headers. Often, this list is too long, and will overlap
%% other information printed in the page headers. This command allows
%% the author to define a more concise list
%% of authors' names for this purpose.
\renewcommand{\shortauthors}{Massetti.}

%%
%% The abstract is a short summary of the work to be presented in the
%% article.
\begin{abstract}
Blockchain is a revolutionary technology that enables users to communicate in a trust-less manner. 
The most prominent change brought by this technology is the mode of business between organizations: they do not need anymore a trusted third party. 
It is a distributed ledger technology based on a decentralized peer-to-peer (P2P) network. 
Since Bitcoin was deployed, many blockchain systems have been born with more capabilities, which have allowed them to fit many
different use cases. Smart Contracts, which are programs running on blockchain systems, could extend the potentiality of blockchain
from a platform for financial transactions to an all-purpose utility.
The development of innovative and prominent applications is a consequence of them, such as NFT marketplaces, music royalty tracking, supply chain and logistics monitoring, voting mechanism, 
cross-border payments, and many others.
Finding bugs and vulnerabilities in them is necessary for assuring their correct behaviour. 
This paper deals with the way for finding the vulnerabilities in Ethereum blockchain-based smart contracts. We review related works regarding 
the classification of the most common vulnerabilities and tools which support their detection of them.
\end{abstract}

%%
%% The code below is generated by the tool at http://dl.acm.org/ccs.cfm.
%% Please copy and paste the code instead of the example below.
%%
\begin{CCSXML}
<ccs2012>
 <concept>
  <concept_id>10010520.10010553.10010562</concept_id>
  <concept_desc>Computer systems organization~Embedded systems</concept_desc>
  <concept_significance>500</concept_significance>
 </concept>
 <concept>
  <concept_id>10010520.10010575.10010755</concept_id>
  <concept_desc>Computer systems organization~Redundancy</concept_desc>
  <concept_significance>300</concept_significance>
 </concept>
 <concept>
  <concept_id>10010520.10010553.10010554</concept_id>
  <concept_desc>Computer systems organization~Robotics</concept_desc>
  <concept_significance>100</concept_significance>
 </concept>
 <concept>
  <concept_id>10003033.10003083.10003095</concept_id>
  <concept_desc>Networks~Network reliability</concept_desc>
  <concept_significance>100</concept_significance>
 </concept>
</ccs2012>
\end{CCSXML}

\ccsdesc[500]{Computer systems organization~Embedded systems}
\ccsdesc[300]{Computer systems organization~Redundancy}
\ccsdesc{Computer systems organization~Robotics}
\ccsdesc[100]{Networks~Network reliability}

%%
%% Keywords. The author(s) should pick words that accurately describe
%% the work being presented. Separate the keywords with commas.
\keywords{Solidity, Software, Vulnerability, Blockchain}


%% A "teaser" image appears between the author and affiliation
%% information and the body of the document, and typically spans the
%% page.


%%
%% This command processes the author and affiliation and title
%% information and builds the first part of the formatted document.
\maketitle

\section{Introduction}
Nowadays, the major platform for decentralized decentralized finance (DeFi) 
and applications (dApps) is Ethereum. It can be described as the "internet of Blockchain".
Its ecosystem consists of the underlying blockchain, a large
variety of smart contracts deployed on it, a wide range of
valuable assets. 

This growing technology has attaracted many investors, indeed, according \href{https://www.coingecko.com/}{CoinGeko}, 
the crypto market's value is standing around \$2 trillion.
On the other hand, interest in such a market has grown even among malicious attackers. 
Attacks such as the “Parity Wallet Hack” and the “Decentralized Autonomous Organization Attack” cost millions of dollars simply because of 
naive bugs in the smart contract code. Blockchain and smart contract technologies have multiple aims, but unfortunately, new applications 
based on them still contain bugs and multiple vulnerabilities, which cause 
several issues for the end-users. Most of the use of this technology relates to finance or certifications, therefore integrity, 
authentication and authorisation in transactions are mandatory. 

The research field behind blockchain technology is growing, as well as the one concerning 
its security and accordingly, many analysis tools were developed. 
These incorporate various strategies for performing the analyses, concerning the technical aspects of smart contracts, 
so these would work differently according to the object of the analysis. 

Among the many aspects of smart contract, our systematic
literature review focuses on studies related to vulnerabilities and 
analysis tools for their detection. We will try to give an answer to the following 
research questions:
\begin{itemize}
    \item Which are the main vulnerabilities in Smart Contracts?
    \item Which methodologies are implemented by analysis tools?
    \item How should we behave for the detection of Vulnerabilities?
\end{itemize}

In Section \ref{RelatedWorks}, we compare the actual papers and works regarding this topic. 
Section \ref{Vulnerabilities} explains the objective of our analysis: Smart Contracts Vulnerabilities. 
We give a taxonomy for the main vulnerabilities regarding Solidity. The classification of analysis tools 
is shown in Section \ref{tools}. We discuss about the main strategies implemented by those.
Options
\begin{itemize}
  \item In the last Section \ref{Conclusion} we try to define a guideline for detecting vulnerabilities. 
  \item In the last Section \ref{Conclusion} we give an overview, defining the suitable cases which 
the tools work better.
  \item In the last section \ref{Conclusion} we propose a real case.
\end{itemize}



\section{Releted Works}
\label{RelatedWorks}
\begin{itemize}
  \item Papers about vulenrabilities detection for defing a taxonomy
  \item papers regarding comperison between tools
  \item papers of the tools that we want to have a Look
\end{itemize}
\citet{SystematicReviewVuln} 

\section{Vulnerabilities in Solidity}
\label{Vulnerabilities}
\section{Security analysis tools}
I create a table with the most common vulnerabilities like reentrancy, arithmetic operations, 
DOS, self distruction. Post some codes.
\label{tools}
I explain the most typologies of tools.
Tools with and without specification.
Fuzzers, symbolic analysis, formal specification.
Level of abstraction of the tools.
\section{Conclusion}
\label{Conclusion}
I would propose 3 options:
\begin{itemize}
  \item propose a real case of analysis: I select a real wolrd exploit and I use the tools for analyses it.
  \item give a guideline for developers, how they should behave. So having a look of vulenrabilities and using the cited tools.
  \item an overview of the tools relating with the vulnerabilities.
\end{itemize}


\begin{acks}

\end{acks}

%%
%% The next two lines define the bibliography style to be used, and
%% the bibliography file.
\bibliographystyle{ACM-Reference-Format}
\bibliography{sample-base}

%%
%% If your work has an appendix, this is the place to put it.
\appendix

\section{Research Methods}

\subsection{Part One}


\end{document}
\endinput
%%
%% End of file `sample-sigconf-i13n.tex'.
